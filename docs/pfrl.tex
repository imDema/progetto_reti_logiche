\documentclass[a4paper,twocolumn]{article}
% \pagestyle{empty}

\title{Progetto finale Reti Logiche}
\author{Luca De Martini}
\date{}

% \usepackage[margin=2cm]{geometry}
\usepackage[hidelinks]{hyperref}
\usepackage[table]{xcolor}
\usepackage{array,multicol}
\usepackage[activate=true,final,tracking=true,kerning=true,spacing=true]{microtype}
\microtypecontext{spacing=nonfrench}

\begin{document}

% \setlength\parindent{0pt}
% \hyphenpenalty=1500
\emergencystretch 3em

\maketitle
% \begin{multicols*}{2}
    
\section{Introduzione}

\section{Architettura}
Al livello più alto il componente presenta un registro di cache dell'indirizzo in output e i process che descrivono il comportamento di una macchina a stati finiti.
La macchina ha 6 stati ed è implementata con dei registri che commutano sul fronte di discesa del clock e ha come ingresso il segnale \texttt{o_enc_rdy}, segnale che il modulo interno porta a \texttt{1} quando l'encoder è pronto a codificare un indirizzo. 
\section{Risultati sperimentali}

\section{Simulazioni}

\section{Conclusioni}

% \end{multicols*}
\end{document}